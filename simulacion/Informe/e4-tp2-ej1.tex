% !TeX root = e4-tp2-ej1
\documentclass[e4-tp2-main.tex]{subfiles}

\begin{document}


\section{Modulaci\'on PWM y realimentaci\'on}


\subsection{Amplificador de error}

\subsubsection{Valores de $R_2$ y $R_3$}
Se pide que la tensi\'on de salida sea $V_{O}=25$V, con lo cual para esta condici\'on tiene que cumplirse $V_{FB}=V_{REF}$. Considerando que para continua, $V_{FB}$ es un divisor de tensi\'on de $V_{O}$, obtiene entonces:

\begin{equation}
	V_{FB} = V_{REF} = V_{O}\cdot \frac{R_3}{R_2+R_3}
\end{equation}

Despejando para $R_2$:

\begin{equation}
	R_2 = R_3 \cdot \left(1+ \frac{V_O}{V_{REF}}\right)
	= R_3 \cdot \left(1+ \frac{25\si{\V}}{2.5\si{\V}}\right)
	= 9R_3
\end{equation}

Dejando $R_3=10\si\kohm$, se obtiene $R_2=90\si\kohm$ (si se quisiese armar el circuito, se podr\'ian poner una resistencia de 68$\si\kohm$ en serie con una de 22$\si\kohm$ sin afectar la precisi\'on). 


\subsubsection{Transferencia para peque\~nas variaciones de $v_o$}

Para la transferencia de peque\~nas se\~nales, podemos pasivar $V_{REF}$ porque no afectara al cambio que se produzca. $R_3$ queda cortocircuitada, y el circuito resultante es un inversor con $Z_1 = R_2$ y $Z_2= R_6+\nicefrac{1}{s \, R_6 C_2}$. Por lo tanto:

\begin{equation}
	\frac{\tilde{v_{C}}}{\tilde{v_O}}(s) = - \frac{Z_2}{Z_1}
	= -\frac{R_6}{R_2} \cdot \left(
		\frac{s+\nicefrac{1}{R_6 C_2}}{s}
	\right)
\end{equation}


\subsubsection{Amplificador de error como bloque de un sistema LTI}

\subsubsection{Fuente de corriente $I_1$ y $R_7$}



\subsection{Modulador PWM}
\subsubsection{Caracter\'isticas de la se\~nal triangular}
La se\~nal definida en la consigna no es una se\~nal perfectamente triangular, puesto que para eso deber\'ia cumplirse $T_{ON}=T_{OFF}=0$. La se\~nal comienza en $V_{INITIAL}=0\si\volt$, sube linealmente a $V_{ON}=19$V en $T_{RISE}=19\si{\micro\s}$, donde se mantiene constante por $T_{ON}=0.25\si{\micro\s}$, para luego bajar en $T_{FALL}=0.5\si{\micro\s}$ a 0V, donde permanece por los restantes 0.25$\si{\micro\s}$ para completar el per\'iodo de $T_{S}=20\si{\micro\s}$.

 
\subsubsection{Duty cycle m\'aximo}

\subsubsection{Modulador PWM como bloque de un sistema LTI}



\subsection{Convertidor DC/DC}
\subsection{Transferencia del convertidor}
Se consideran ideales al diodo y al MOS.

\subsubsection{Valor real del duty cycle}


\subsubsection{Tiempo de establecimiento}


\subsubsection{title}

\end{document}
